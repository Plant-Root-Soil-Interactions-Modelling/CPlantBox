\documentclass[a4paper]{article}

\usepackage[latin1]{inputenc} 
\usepackage{graphicx}
\usepackage{natbib} 
\usepackage{amsmath}
\usepackage{subcaption}

\usepackage{listings}
\usepackage{color}
\usepackage{natbib}

\usepackage[hidelinks]{hyperref}

\newcommand{\bm}[1]{\mbox{\boldmath{$#1$}}}

\definecolor{codegreen}{rgb}{0,0.6,0}
\definecolor{codegray}{rgb}{0.5,0.5,0.5}
\definecolor{codepurple}{rgb}{0.58,0,0.82}
\definecolor{backcolour}{rgb}{0.95,0.95,0.92}

\lstdefinestyle{mystyle}{
    backgroundcolor=\color{backcolour},   
    commentstyle=\color{codegreen},
    keywordstyle=\color{magenta},
    numberstyle=\tiny\color{codegray}, 
    stringstyle=\color{codepurple},
    basicstyle=\footnotesize,
    breakatwhitespace=false,         
    breaklines=true,                 
    captionpos=b,                    
    keepspaces=true,                 
    numbers=left,                    
    numbersep=5pt,                  
    showspaces=false,                
    showstringspaces=false,
    showtabs=false, 
    tabsize=2
} 
  
\lstset{style=mystyle}
 


\begin{document}

\begin{center}
%\includegraphics[width=0.2\textwidth]{sw_logo} \\
\vspace{0.5 cm}
\huge{\textbf{CPlantBox Tutorial}} \\
\vspace{0.5 cm}
\normalsize
Daniel Leitner, Andrea Schnepf, et al. \\
\end{center}

\vspace{0.5 cm}

\noindent 
The following tutorial offers scripts to outline the usage of the CPlantBox \citep{zhou2020cplantbox} and its Python binding which is named \emph{plantbox}. CPlantBox was developed from CRootBox \citep{schnepf2018crootbox} and is largely backward compatible by having the same underlying rootsystem model. For further documentation please refer to the Doxygen class documentation of the CPlantBox code.
 

\vspace{0.5 cm}

\tableofcontents

\newpage
\section{Introduction} \label{sec:introduction}

\subsection{Installation} \label{ssec:installation}
\input{latex/intro_installation.tex}

\subsection{Basic example} \label{ssec:basic_example}
To test the installation first go to the examples \lstinline{cd tutorial/examples/} and run the following 'Hello World' like example by \lstinline{python3 intro_basic.py}. It shows a typical CPlantBox simulation and Figure \ref{fig:intro_basic} shows the simulation result. The main steps include: open a CPlantBox parameter file (L13), perform the simulation (L20), save the results (L23-L26), and make an interactive plot showing the results (L29):  

\lstinputlisting[language=Python, caption=Basic example (intro\_basic.py)]{examples/intro_basic.py} 

\noindent 
Lets revise the above code in more detail: 
\begin{itemize}
 \item[2] We add the path to find the \emph{plantbox} module.
 \item[4] Imports the CPlantBox Python library \emph{plantbox} and name it pb.
 \item[5] Imports a auxiliary script for visualization of the rootsystem with VTK and name it vp.
 \item[8] Constructs the plant object.
 \item[12] Opens an .xml containing parameters describing the structural properties of the plant by defining the seed (SeedRandomParameters), the stem (StemRandomParameters), the leaf (LeafRandomParameters) and the roots (RootRandomParameters). Alternatively, all parameter can be set or modified directly in Python. A more detailed description is given in Section \ref{sec:cplantobx}.
 \item[16] Initializes the simulation: Creates the initial stem, tap root and the base roots
 (i.e. all basal roots, and shoot borne roots that might emerge). Initializes the tropisms and passing the domain geometry, 
 and creates the elongation functions. 
 \item[20] Performs the simulation. The value 40 is the simulation time in days. If no simulation time is passed the simulation time is taken from the parameter file. Note that simulation results are independent from the time step, i.e. 40 simulate(1) calls should yield a similar result 
 as simulate(40) (due to stochasticity we cannot expect the exact same result). 
 \item[23] Saves the resulting plant geometry in the VTK Polygonal Data format (VTP) where each plant organ is repesented as a polylines. Use Paraview to visualize the results, see Section \ref{ssec:visualisation}.
 \item[25,26] If we want to visualize simulation results that are given per segment another option is to export the plant geometry segment wise. L25 creates a segment based representation of the plant geometry, and L26 saves it as VTP file.
 \item[29] Create an interactive plot (use mouse to rotate or zoom) using VTK. Per default plant 'age', 'creationTime', 'radius','organType' or 'subType' can be viusalized. The organType is 1 for the seed, 2 for root, 3 for stem, and 4 for leafs. The subType is the number of the parameter set within a organ class, i.e. the type of root, stem or leaf. You can rotate (left click) pan (right click) or zoom (mouse wheel) and save a screenshot as png file by pressing 'g', or reset view 'r', or change view by pressing 'x', 'y', 'z', and 'v'.
 \end{itemize}
  
\begin{figure}
\centering
\includegraphics[width=0.5\textwidth]{figures/intro_basic.png} 
\caption{Plant after 40 days simulation time.} \label{fig:intro_basic}
\end{figure}  




\newpage
\section{Topics / Applications} \label{sec:topcis}

This section describes the usage of CPlantBox by small examples wihtout going into details of the underlying model. A more precise description of the structural model is given in following Section \ref{sec:cplantobx}. 

\subsection{Development over time} \label{ssec:development}

The following script shows how to analyse root system length versus time. 

\lstinputlisting[language=Python, caption=Organ development over time]{examples/topics_development.py}

\begin{itemize}
\item[9-14] Sets up the simulation.

\item[16-18] Defines the simulation time, time step, and the resulting number of simulate(dt) calls. 

\item[21] First we state which scalar type we want to analyse ('volume', 'surface', or 'one' would also make sense). 

\item[22] Pre-definition of the numpy arrays storing the lengths over time. 

\item[23-30] The simulation loop executes the simulation for a single time step L24. L25 calculates the type of each root (the organ subType), L26 the length (or any other parameter) of the root. L27-L30 calculates the total root length at the current time step for all roots, and for specific root types. The method rs.getParameter collects this data from the RootSystem organs. It is possible to access all root random parameters and resulting realisations using rs.getParameter. In C++ the class functions are defined in Root::getParameter, Organ::getParameter. 

\item[32-38] Creates Figure \ref{fig:length}.

\end{itemize}



Next we show two options how to retrieve root tip and root base positions from a simulation:

\lstinputlisting[language=Python, caption=Organ tips and organ bases over time]{examples/topics_development2.py}

\begin{itemize}

\item[14,15] Reset the simulation and simulate for only 7 days (otherwise there are so many root tips).

\item[17-18] Outputs the number of nodes and segments to get an idea how big the resulting root system is. 
Note that number of segments equals the number of nodes minus the number of base roots that will emerge. 
Base roots are tap roots, basal roots and shootborne roots.

\item[20-26] The first approach retrieves all roots as polylines L21. 
Root tips are the last nodes of the polylines L26, root bases the first nodes L25. Roots that have not started to grow have only 1 node, and are not retrieved by getPolylines().

\item[28-31] Second approach: L29 rs.getNodes() returns all nodes of the root system as a list of nodes, i.e. Vector3d objects. Each Vector3d object can be converted into a numpy array automatically, but is necessary to do that for each element of the list. The methods L30, L31 return the indices of the tips and bases. 

\item[33-41] Creates Figure \ref{fig:scatter} using the second approach.

\item[44,45] Verifies that both approaches yield the same result.

\end{itemize}

\begin{figure}
\begin{subfigure}[c]{0.5\textwidth}
\includegraphics[width=0.99\textwidth]{examples/results/topics_development.png}
\subcaption{Total organ lengths versus time} \label{fig:topics_development}
\end{subfigure}
\begin{subfigure}[c]{0.5\textwidth}
\includegraphics[width=0.99\textwidth]{examples/results/topics_development2.png}
\subcaption{Top view of the organ tip and bases} \label{fig:topics_development2}
\end{subfigure}
\caption{Plant development over time} 
\end{figure}




\subsection{How to virtually mimic experiments} \label{ssec:virtual}
% In order to mimic experimental settings we can confine root growth by containers, or we can implement obstacles hindering root growth. Furthermore, periodic domains can be used to mimic field conditions. In CPlantBox the domain geometry is represented in a mesh free way using signed distance functions (SDF). A SDF returns the distance of a point to its closest boundary, with negative sign if it lies within the domain, and a positive if the point is outside of the domain. CPlantBox has auxiliary functions for creating simple domains, which is shown in the following example.

\subsubsection*{Growth in a container}

We show two examples where the plants root system grows confined by two types of containers, a cylindrical container or a rectangular rhizotron. 

\lstinputlisting[language=Python, caption=Root growth in a container (topics\_virtual.py)]{examples/topics_virtual.py}

The geometry is first created by constructing a specialization of the class {SignedDistanceFunction}, 
which is passed to the root system by the method {plant.setGeometry()}: 
\begin{itemize}
 \item[9-11] Choose the parameter input file
 \item[14] Construct a cylindrical container. 
 \item[17] Construct a rhizotron.
 \item[20] Pick one of the two geometries. Note that it is important to call plant.setGeometry() before plant.initialize().
 \item[23,24] Initializes and simulates for 40 days.
 \item[27] Exports the plant structure geometry (without the soil domain geometry).
 \item[31] Its possible to save the soil domain geometry as Paraview Python script for visualization (and debugging), see Figure \ref{fig:topics_virtual}. Run this script in Paraview by Tools$\rightarrow$Python Shell, Run Script. 
\item[34] Interactive VTK plot. The geometric boundaries can currently not be visualized in the interactive rendering. This could be achieved in VTK by creating an iso-surface of the implicit geometry given by the SDF. visualised. 
\end{itemize}

\begin{figure}
\begin{subfigure}[c]{0.5\textwidth}
\includegraphics[width=0.99\textwidth]{figures/topics_virtual_a.png}
\subcaption{Confined by a cylinder (colour represents root age)} 
\end{subfigure}
\begin{subfigure}[c]{0.5\textwidth}
\includegraphics[width=0.99\textwidth]{figures/topics_virtual_b.png}
\subcaption{Confined by a rhizotron (colour represents root type)} 
\end{subfigure}
\caption{ParaView visualizations of results.} \label{fig:topics_virtual}
\end{figure}

Next, we show how to build more complex container geometries using SDF. 


\subsubsection*{Complex containers using SDF with set operations}

In the following example we create some geometries that we might encounter in actual experiments. First, we show how to rotate a rhizotron (e.g. to see more roots at the wall due to gravitropism). Second, we create a split box experiment, and furthermore, an example where rhizotubes act as obstacles. The following examples demonstrates how to build a complex geometry using rotations, translations and set operations on the SDF.

\lstinputlisting[language=Python, caption=Root growth in more complex containers (topics\_virtual2.py)]{examples/topics_virtual2.py}

\begin{itemize}
\item[14-19] Definition of a rotated rhizotron, see Figure \ref{fig:topics_virtual2_a}: 
L15 creates the flat container with a small height, this container is then rotated and translated into the desired position. L16 is the location of the plant seed within the unrotated rhizotron. L17 defines the rotational matrix rotating around the x-axis. In L18 the seed position is rotated. Finally, in L21 the rhizotron is rotated and translated so that the seed location is moved to the origin. 
\item[21-30] Definition of of a split box, see Figure \ref{fig:topics_virtual2_b}: 
The split box is composed of a left box, a right box, and a top box connecting left and right. 
In L30 the geometry is defined by the set operation union of the three compartments. 
\item[33] Pick one of the three geometries for your simulation.
\item[39] Also more complex geometries can be visualized by the Paraview script, 
however, set operations are not really performed, only the involved geometries are visualized.
\item[40] We cannot visualize the container geometry in the interactive rendering, but only the resulting root system. 
\end{itemize}

\begin{figure}
\begin{subfigure}[c]{0.49\textwidth}
\includegraphics[width=0.99\textwidth]{figures/topics_virtual2_a.png} 
\subcaption{Rotated rhizotron} \label{fig:topics_virtual2_a}
\end{subfigure}
\begin{subfigure}[c]{0.49\textwidth}
\includegraphics[width=0.99\textwidth]{figures/topics_virtual2_b.png} 
\subcaption{Split box experiment} \label{fig:topics_virtual2_b}
\end{subfigure}
\caption{Complex container geometries described by SDF and set operations.}
\end{figure}

\subsubsection*{Obstacles using SDF}

We can also use set operations to create obstacles. The following example shows a rhizotube camera setup, where transparent tubes are used to analyse root growth. We can mimic this setup by defining tubes that act as an obstacle to the growing roots. The following code is similar to before, but using another geometry: 

\lstinputlisting[language=Python, caption=Experimental setup with rhizotubes (topics\_virtual3)]{examples/topics_virtual3.py}

Definition of rhizotubes as obstacles, see Figure \ref{fig:topics_virtual3}:
\begin{itemize}
\item[14] Defines the surrounding box 
\item[15,16] Definition of a single rhizotube, that is rotated around the y-axis. 
\item[21,26] Create a list of rhizotubes at different locations that mimics the experimental setup.  
\item[27,28] Composes the final geometry by two set operation: first a union of all tubes, and then cut them out the surrounding box by taking the difference. 
\end{itemize}

\begin{figure}
\centering
\includegraphics[width=0.7\textwidth]{figures/topics_virtual3.png}
\caption{Rhizotubes act as obstacles to the root system} \label{fig:topics_virtual3}
\end{figure}

\subsubsection*{Multiple root systems}

Its possible to simulate multiple root systems. In the following we show a small plot scale simulation:

\lstinputlisting[language=Python, caption=Multiple root systems (topics\_virtual4.py)]{examples/topics_virtual4.py}

\begin{itemize}
\item[11,12] Set the number of columns and rows of the plot, and the distance between the root systems.
\item[15-24] Creates the root systems, and puts them into a list \texttt{all}. L20 retrieves the plant seed, and L21 sets a new seed position. 
\item[26,27] Simulate all root systems.
\item[30-37] Saves each individual root systems, and additionally, saves all root systems into a single file. 
Therefore, we create an SegmentAnalyser object (see Section \ref{ssec:postprocessing}) in L30 and merge all organ segments into it (L34). Finally, we export a single VTP file (L37). The resulting geometry is shown in Figure \ref{fig:topics_virtual4}, where ParaView was used for visualisation (see Section \ref{ssec:visualisation} - ParaView).
\end{itemize}

\begin{figure}
\centering
\includegraphics[width=0.7\textwidth]{figures/topics_virtual4.png}
\caption{ParaView visualization of multiple root systems.} \label{fig:topics_virtual4}
\end{figure}


\subsubsection*{Periodic domains}

If we consider only one plant type we often simplify field scale simulations to a single plant simulation with a periodic domain where the domain length and width is determined by planting density and inter-row distance. We can analyse the root geometry mapped to a periodic grid using SegmentAnalyser.mapPeriodic(), see Section \ref{ssec:postprocessing}. For coupling a root system with a periodic macroscopic soil model an unimpeded single root system is calculated and mapped into the periodic domain by the root to soil mapping function, see Section \ref{ssec:mapped}.





\subsection{Visualisation} \label{ssec:visualisation}
% \subsubsection*{Quick visualizations from Python}

interactive plots...

\begin{lstlisting}[language=Python]
vp.plot_plant(plant)
\end{lstlisting}

\subsubsection*{How to export plant geometry for later visualization} \label{ssec:export}

There are various ways to export plant geometry for later visualization, or to use the geometry as mesh for other simulation software.

The Plant class provides outputs in RSML (), VTP (), and for paraview python scripts for the soil container geometry. Each organ is represented as a polyline with additional data given for each organ/polyline.

\begin{lstlisting}[language=Python]
plant.write("myfile.rsml")
plant.write("myfile.vtp")
plant.write("myfile.py")
\end{lstlisting}

The SegmentAnalyser class provides outputs in VTP (), DGF (), and text files. The Plant organs are represented by their segments.
\begin{lstlisting}[language=Python]
ana = pb.SegmentAnalyser(plant)
ana.write("myfile.vtp")
ana.write("myfile.dgf")
\end{lstlisting}


The interactive plotting tools are based on VTK and it is easy to write the resulting geometries as VTP in a binary format resulting in smaller file size. The organs are represented as polylines, the
\begin{lstlisting}[language=Python]
vp.write_plant(plant)
\end{lstlisting}
TODO there are file formats supported, e.g. mesh


The class PlantVisualizer can ...




\subsubsection*{Paraview}

how to run the macros... and render the geometry


\subsubsection*{How to make an animation} \label{ssec:animation}

In order to create an animation in Paraview we have to consider some details. The main idea is to export the result file as segments using the class SegmentAnalyser. A specific frame is then obtained by thresholding within Paraview using the segments creation times. In this way we have to only export one vtp file. 

We modify example1b.py to demonstrate how to create an animation.

\lstinputlisting[language=Python, caption=Example 4c (modified from Example 1b)]{examples/example3e_animation.py} 

\begin{itemize}

\item[11,12] Its important to use a small resolution in order to obtain a smooth animation. L18 set the axial resolution to 0.1 cm. 

\item[19,29] Instead of saving the root system as polylines, we use the SegmentAnalyser to save the root system as segments.

\item[22,23] It is also possible to make the root system periodic in the visualization in $x$ and $y$ direction to mimic field conditions.

\item[26-28] We save the geometry as Python script for the visualization in ParaView.

\end{itemize}

After running the script we perform the following operations Paraview to create the animation:
\begin{enumerate}
 \item Open the .vtp file in ParaView (File$\rightarrow$Open...), and open tutorial/examples/python/results/example\_3e.vtp.
 \item Optionally, create a tube plot with the help of the script tutorial/pyscript/rsTubePlot.py (Tools$\rightarrow$Python Shell, press 'Run script').
 \item Run the script tutorial/pyscript/rsAnimate.py (Tools$\rightarrow$Python Shell, press 'Run script'). The script creates the threshold filter and the animation. 
 \item Optionally, visualize the domain boundaries by running the script tutorial/examples/python/results/example\_4e.py (Tools$\rightarrow$Python Shell, press 'Run script'). Run after the animation script (otherwise it does not work).  
 \item Use File$\rightarrow$Save Animation... to render and save the animation. Pick quality ($<$100 \%), and the frame rate in order to achieve an appropriate video length, e.g. 300 frames with 50 fps equals 6 seconds. The resulting files might be uncompressed and are very big. The file needs compression, for Linux e.g. ffmpeg -i in.avi -vcodec libx264 -b 4000k -an out.avi, produces high quality and tiny files, and it plays with VLC.
\end{enumerate}



\subsection{Postprocessing} \label{ssec:postprocessing}
% The class SegmentAnalyser offers post-processing methods based on a representation of the plant by segments which connect two nodes with a single line. The segments are derived from the centerlines of the plant organs. We can use this approach to do distributions or calculate densities, and we can analyse the segments within any geometry by cropping the overall geometry to a region of interest. While the class SegmentAnalyser was designed for analysing roots most functionallity works also for the above plant part. 

\subsubsection*{Root surface densities}

We start with a small example plotting the root surface densities of a root system versus root depth.

\lstinputlisting[language=Python, caption= Calculating root surface densities in a soil column]{examples/topics_postprocessing.py}

\begin{itemize}
\item[9-12] Pick a plant or root system.
\item[14-16] Depth is the length of the soil column (into z-direction), layers the number of vertical soil layers, where the root surfaces are accumulated, and runs is the number of simulation runs. 
\item[18] Creates a confing geometry.
\item[20-26] Performs the simulations for $runs$ times. L26 creates a distribution of a parameter (name) over a vertical range (bot, top). The data are accumulated in each layer, segments are either cut (exact = True) or accumulated by their mid point (exact = False). 
\item[28,29] In order to calculate a root surface density from the summed up surface, we need to define a soil volume. The vertical height is the layer length, length and width (here 10 cm), can be determined by planting width, or by a confining geometry. 
\item[30, 31] Calculates the densities mean and the standard error. 
\item[33-42] Prepares the plot (see Figure \ref{fig:topics_postprocessing}).
\end{itemize}

\begin{figure}
\centering
\includegraphics[width=0.7\textwidth]{figures/topics_postprocessing.png} 
\caption{Root surface densitiy versus depth including standard deviation (based on 10 simulation runs) } \label{fig:topics_postprocessing}
\end{figure}




\subsubsection*{Analysis of segment within a cropping geometry}

The following script demonstrates some of the post processing possibilities by setting up a virtual soil core experiment (see Figure \ref{fig:soilcoreGeom}), where we analyse the content of two soil cores located at different positions.

\lstinputlisting[language=Python, caption= A virtual soil core experiment (topics\_postprocessing2)]{examples/topics_postprocessing2.py}

\begin{itemize}
\item[11-15] Performs the simulation.
\item[17-22] We define two soil cores, one in the center of the root and one 10 cm translated. In L22 we pick which one we use for the further analysis. Figure \ref{fig:soilcoreGeom} shows the resulting geometry, with a soil core radius of 10 cm.
\item[24-28] Prepares three sub-figures. 
\item[31-41] Creates a root length distribution versus depth at different ages. L33 creates the SegmentAnalyser object, and L34 crops it to a fixed domain or maps it into a periodic domain. In L38 the filter function keeps only the segments, where the parameter (first argument) is in the range between second and third argument. L39 creates the distribution. 
\item[44-54] We repeat the procedure, but we crop to the soil core selected in L22. 
\item[57-89] In the third sub plot we make densities of specific root types like basal roots, first order roots, and second order roots. In L58 we crop the segments to the soil core geometry. In L63 we filter for the selected sub type, and in L64 we create the density distribution.
\item[71-73] Show and save resulting Figure \ref{fig:central} and \ref{fig:shifted} for the two soil cores (chosen in L22).
\end{itemize}

% \begin{figure}
% \centering
% \includegraphics[width=0.7\textwidth]{example_3d.png} % 
% \caption{Virtual soil cores experiment (Example 3b): Central core (blue), shifted core (red)} \label{fig:soilcoreGeom}
% \end{figure} 


The example shows differences between the central core and shifted core (see Figure \ref{fig:central} and \ref{fig:shifted}) because the central core captures all roots emerging from the seed. The basic idea is that such analysis can help to increase the understanding of variations in experimental observations.
% 
% \begin{figure}
% \centering
% \includegraphics[width=0.9\textwidth]{example3b.png} 
% \caption{Central core (Example 3b)} \label{fig:central}
% \end{figure}
% 
% \begin{figure}
% \centering
% \includegraphics[width=0.9\textwidth]{example3b2.png} 
% \caption{Shifted core (Example 3b)} \label{fig:shifted}
% \end{figure}


\subsubsection*{SegmentAnalyser for measurements}

It is also possible to make use of the SegmentAnalyser class without any other CPlantBox classes (e.g. for writing vtp from measurements). The following example shows how to construct the class with arbitrary nodes and segments (e.g. from measurements). 

\lstinputlisting[language=Python, caption=SegmentAnalyser can be used on measured data (topics\_postprocessing3.py) ]{examples/topics_postprocessing3.py}

\begin{itemize}
 \item[7-11] Define some segments with data
 \item[14,15] We convert the Python list to lists of C++ types
 \item[18] We create the SegmentAnalyser object without an underlying plant
 \item[19,20] Use the Analyser object, by printing information, or writing a vtp. 
 \item[21] Visualize your results
\end{itemize}


\subsection{Playing with parameters} \label{ssec:parameters}
% 

\subsubsection*{Initialize parameters from scratch} \label{sec:from_scratch}
 
In the previous examples we opened the root system parameters from an xml file. In this example we show how to do everything in a Python script without the need of any parameter file. This is especially important if we want to modify parameters in our scripts (e.g. like it is needed for a sensitivity analysis, see Section \ref{ssec:sensitivity}).

In order to set up a simulation by hand, we have to define all relevant model parameters. This is done by creating a RootRandomParameter object for each root order or root type, and one SeedRandomParameter for each plant type. 

Note that during the simulation, the parameters for a specific root (RootSpecificParameter) are generated from the RootRandomParameter class which represents the random distributions of certain parameters.

\lstinputlisting[language=Python, caption=Example 2a]{examples/example2a_initializeparams.py}

\begin{itemize}
\item[5] Matplotlib is Python's easy way to create figures like in Matlab.\item[6] NumPy is Python's scientific computing package.

\item[9,10] Create the root type parameters of type 1 and type 2.
\item[12-38] We set up a simple root system by hand. First we define the tap root L12-L26, then the laterals L28-L38. By default all standard deviations are 0. Most parameters standard deviations can be set with an additional 's' appended to the parameter name, e.g. $lmaxs$ is the standard deviation of $lmax$, see L32
\item[40,41] Set the root type parameters.

\item[43-47] Create an object of class SeedRandomParameter which defines when basal and shoot borne roots emerge. In this example we neglect basal and shoot borne roots, and just define the seed location, and deactivate basal roots by setting their maximal number $maxB$ to 0 ($firstB$ and $delayB$ are ignored in that case). 
\item[48] Sets the root system parameters.

\item[53] We choose the simulation times in a way that we can see the temporal development, and that all lateral roots have emerged in the final time step.

\item[54-70] Within the simulation loop we create Figure \ref{fig:ip}. L58-61 defines the limits and titles. In L63 we retrieve the roots as polylines which are represented by a list of nodes. In L64-67 we plot the $x$ and $z$ coordinates for each segment ($n$, $n2$) as green line. 

\item[75-80] It is not only possible to set all model parameter, 
but to retrieve the parameters after the simulation with rs.getParameter(), which returns one value per root. For all parameters that are derived from a random distribution the root specific parameter is returned (e.g. $la$, L78), i.e. the values that were drawn from the normal distribution. The root random parameter can be accessed by adding '\_mean', '\_dev' to the parameter value (e.g. $la\_mean$, L79).

\end{itemize}

Note that all parameters can be set and modified within Python. Especially, standard deviations can be set to zero in order to be able to precisely predict the result. For example we can calculate the total root system length analytically, and check if the numerical simulation yield the (exact) same result. This is performed in the tests test\_root.py, and test\_rootsystem, which is used to test and validate CPlantBox (see folder CPlantBox/test).

With such simple simulations, we can quickly check if the model does, what we expect. For example the maximal number of laterals of above parameters is 16 
$= round(lmax - la - lb)/l_n +  1$. We can calculate the time when the final lateral emerges as $-(lmax/r)*\ln(1-(lmax-l_n/2)/lmax)$ = 122.8 days. At simulation time 125 the last lateral root that has emerged is 2.2 days old, and therefore approximately 4.4 cm long (initial growth rate $r_1 = 2$), which agrees with Figure \ref{fig:ip}.

By default the length of the apical zone is fixed, when the root is created. During growth the apical zone stays in the interval $[la - l_n/2, la+l_n/2]$. The first branch emerges at length $lb$, when the root length reaches $lb +la$.

In the following two subsections we show, how tropism parameters and inter lateral spacing will affect the resulting root system.

% \begin{figure}
% \centering
% \includegraphics[width=\textwidth]{fig_initializeparams.png}
% \caption{Root development} \label{fig:ip}
% \end{figure}

\subsubsection*{Root tropism parameters $N$ and $\sigma$} \label{ssec:tropism}

We show the influence of the parameters $N$ and $\sigma$ in the case of gravitropism. The parameter $N$ [1] denotes the strength of the tropism, and $\sigma$ [cm$^{-1}$] the flexibility of the root, i.e. the expected angular variation per cm root length. 

\lstinputlisting[language=Python, caption=Example 2b]{examples/example2b_tropism.py}

\begin{itemize}
\item[10-13] We choose the parameter values $N$ and $\sigma$ we want to plot. It might be interesting to change the initial insertion $angle$, and the axial resolution $dx$. Note that a change in axial resolution should not qualitatively change the resulting images.

\item[15-18] The root system is created and SeedRandomParameters are set to produce a basal root every third day. 

\item[20-23] The RootRandomParameter for tap and basal roots is defined.

\item[25-28] The loop runs over the parameters we want to modify. We create a subplot for each configuration, and start with a new root system by calling reset in L28.

\item[30-34] We set the parameters (L30,L31) and do the simulation (L33,L34)

\item[36-44] Matplotlib is used for visualization (looping over the segments is rather slow). L36 gives a list of all nodes, and L37 of all segments as two node indices. Therefore, each segment starts at node n1 and ends at node n2, as defined in L39.

\item[L48] Creates Figure \ref{fig:tropism}.
\end{itemize}

In the first column ($\sigma$=0) of Figure \ref{fig:tropism} nothing happens, because if the root has no flexibility to bend, the strength has no influence on the resulting root system. The first row ($N$=0) shows the influence of $\sigma$ only. The growth is undirected, because the strength of the gravitropism is zero. If the roots have small flexibility, they grow downwards because they initially do. 

The other subplots show different shapes of the root system. We normally derive the two parameters $N$ and $\sigma$ by visual comparison. Note that results are independent of the root axial resolution $dx$ and the temporal resolution.
% 
% \begin{figure}
% \centering
% \includegraphics[width=\textwidth]{fig_gravitropism.png}
% \caption{Influence of $N$ and $\sigma$ in case of gravitropism} \label{fig:tropism}
% \end{figure}




\subsubsection*{Inter lateral spacing ($ln$, $lnk$)} \label{ssec:spacing}

A single root is divided in basal zone, root branching zone, and root apical zone. Basal and apical zone are given by the parameters $la$, and $lb$ with standard deviations $la_s$ and $lb_s$. The branching zone has the size $lmax-la-lb$, where $lmax$ is the maximal root length. The branching zone is divided into inter lateral distances $l_n$, which are values drawn from a normal distribution with standard deviation $l_{ns}$. All values are fixed when the root is created This is performed in the method RootRandomParameter::realize(). The chosen parameters reflect the root growth under perfect conditions. Based on this, the root development can then be influenced by environmental conditions e.g. impeding growth speed, or lateral emergence, see Section \ref{sec:functional}.

Normally, the setting constant branching distances is sufficient, but sometime experimental data indicate that inter lateral distances are smaller, or larger near the base than near the root tip. The reason for this could be soil root interaction (e.g. root response to dense or nutritious layer), or within the genotype. We added a purely descriptive parameter to mimic such experimental observations. The parameter $lnk$, which is zero per default, defines the slope, at the mid of the branching, altering the inter lateral distance linearly along the root axis. In the following script, we demonstrate the usage of $lnk$.

\lstinputlisting[language=Python, caption=Example 2c]{examples/example2c_lateralspacing.py}

\begin{itemize}
\item[10-13] A root system with a single tap root is created. 
\item[16,16] The axial resolution, and insertion angle is defined. We take a very large axial resolution for the tap root, since we visualize the nodes later on, and we want to see only lateral branching nodes.
\item[18-24] Definition of the tap root. Standard deviations are zero, we do not want any variations. Tropism parameters are chosen in a way, that the tap root grows straight downwards.
\item[26-29] Definition of the first order lateral. Tropism is a strict exotropism (i.e. root follows its initial growth direction).
\item[31,32] The parameter values we want to visualize.
\item[36] Resets the root system (to $simTime = 0$). Root parameters are not changed. 
\item[38,39] Sets the values for this subplot. 
\item[41,42] Runs the simulation
\item[44-56] Creates the subplots. First (L47-49) we plot all segments in green. And second (L51-53) we plot all nodes of the tap root as red asterisks.
\item[58-60] Creates Figure \ref{fig:spacing}
\end{itemize}

The mid column of Figure \ref{fig:spacing} shows to different inter lateral distances, 4 (top) and 2 (cm) bot. The left column demonstrate the use of negative values for $lnk$ which results in larger distances near the base. The right column has positive values for $lnk$, which will result in smaller distances near the base. The $i$-th inter distance is calculated as $ln_i = ln + lnk (x_i-mid_x)$, where $x_i$ is the position within the branching zone, and $mid_x$ is the mid of the branching zone. This is done in RootRandomParameter::realize(). Note that $lnk$ is dimensionless and the slope in the linear equation. At mid of the branching zone the inter-lateral distance equals $ln$. 

% \begin{figure}
% \centering
% \includegraphics[width=0.99\textwidth]{fig_lateralspacing.png}
% \caption{Inter-lateral spacing, larger near base (left column), constant (mid column), and smaller near base (right column) } \label{fig:spacing}
% \end{figure}

In the following we show, how to analyse model results on a per root basis using the RootSystem class. To create density distributions the resulting root segments are analysed using the SegmentAnalyser class, described in Section \ref{sec:sa}.

\newpage
\subsubsection*{Tropisms} \label{sec:tropism}

The change in root growth direction is described by tropisms. In the following we show how to implement directed growth toward higher water content or nutrient concentration, and demonstrate how to simply make new user defined tropism rules. 


Root growth direction is influenced by soil conditions such as water content, soil strength, or nutrient concentration. In the following example we model the influence of a nutrient rich layer to root system development

\lstinputlisting[language=Python, caption=Example 4a]{examples/example4a_hydrotropism.py}

\begin{itemize}

\item[6-9] Creates the root system and opens the parameter file

\item[12-17] Change the tropism for all root types: L13 modifies the axial resolution, L14 set the tropism to hydrotropism, and L15-16 sets the two tropism parameters. The parameter $\sigma$ is set to 0.4 for the tap root ($subType$ = 1), and to 1. for the rest of the root types.

\item[19-25] Definition of a static soil property using SDF. We first define the geometry (L20-L21), and then create a static soil (L22) that obtains the maximal value $maxS$ inside the geometry, 
$minS$ outside the geometry, and linear slope with length $slope$. At the boundary the soil has the value $(maxS+minS)/2$.

\item[28] Sets the soil. Must be called before RootSystem::initialize()

\item[41] Initializes the root system, and among others sets up the hydrotropism. 

\item[33-39] Simulation loop

\item[42] Exports the root system geometry

\item[45-46] We actually do not wish to set this geometry, but we abuse the writer of the class RootSytem to export a Python script showing the layer geometry. The resulting ParaView visualization is presented in Figure \ref{fig:chemo}.

\end{itemize}

% \begin{figure}
% \centering
% \includegraphics[width=0.7\textwidth]{example4a.png}
% \caption{Chemotropism in a nutrient rich layer (Example 4a)} \label{fig:chemo}
% \end{figure}

Normally, the simulation is created from a set of parameters. For tropisms these are the type of tropism $T$, number of trials $N$ , and tortuosity $\sigma$. There are only a few predefined tropisms called 'gravi', 'plagio', 'exo', or 'hydro', but it is simple to add user defined tropisms.
The following example demonstrates how to define a root age dependent tropism, where roots first grow according to exotropism (following the inital growth direction), and after a certain age change to gravitropic growth.

The new tropism class must be derived from the class Tropism. In CPlantBox tropism is realised with a random optimization process, where the 'best' direction is chosen from $N$ possible direction, according to an objective function that is minimized. Normally, it is sufficient to overwrite this function, called Tropism ::tropismObjective, to change the tropism behaviour. This can be done in Python or in C++. The classes Hydrotropism, Gravitropism, and Plagiotropism (in tropisms.h) are examples for this procedure.

If the whole concept of random optimization is altered, Tropism ::getUCHeading must be overwritten, which is only possible in C++. If the geometry model is also changed Tropism::getHeading must be overwritten.

The following example shows how to implement a new tropism in Python. Two new tropism are introduced:
The first does nothing but to output the incoming arguments of the method Tropism::tropismObjective to the command line (e.g. for debugging). The second one describes a root age dependent tropism that starts with exotropism and changes with root age to gravitropism.

\lstinputlisting[language=Python, caption=Example 4b]{examples/example4b_usertropism.py}

\begin{itemize}

\item[8-19] Creates a new tropism that just writes incoming arguments of Tropism ::tropismObjective to the command line. This can be used for debugging. The new class is extended from rb.Tropism, and the method Tropism ::tropismObjective is overwritten with the right number of arguments.

\item[22-37] Again, we extend the new class from rb.Tropism. In L25-30 we define our own constructor. Doing this two things are important: (a) the constructor of the super class must be called (L26), and (b) the tropism parameters $n$, and $\sigma$ must be set (L29). 
Furthermore, the constructor defines two tropisms: exo- and gravitropism, that are used in Tropism::tropismObjective at a later point, and a root age that dermines when to switch betwee exo- and gravitropism. \\
In L32-L37 the method Tropism::tropismObjective is defined. We choose the predefined objective function depending on the root age.

\item[41-45] Sets up the simulation.

\item[48-51] L48,L49 creates the first user defined tropsim. Since we did not define a constructor Tropism::setTropismParameter must be called. L50 creates the second user defined tropism.  In L51 the tropism is chosen, using the method Tropism::setTropism. The second argument states for which root type it applies. 
Number 4 is the (default) root type for basal roots, -1 states that the tropism applies for all root types (default = -1).

\item[54-58] The simulation loop. 

\item [61] Exports the result producing Figure (\ref{fig:tropism}). 

\item [64] VTK plot.

\end{itemize}

% \begin{figure}
% \centering
% \includegraphics[width=0.7\textwidth]{example4b.png}
% \caption{Depending on root age the laterals follow plagio- or gravitropism (Example 4b)} \label{fig:tropism}
% \end{figure}






\subsection{Sensitivity analysis} \label{ssec:sa}
% 
In the next part we vary given parameters in order to make a sensitivity analysis. This takes a lot of simulation runs, and we demonstrate the of use parallel computing to speed up execution.

In this exemplary sensitivity study we will vary the insertion angle of the tap and basal root, and look at the resulting change in mean root tip depth and root tip radial distance. 

\lstinputlisting[language=Python, caption=Example 2f]{examples/example2f_sensitivity.py}

\begin{itemize}

\item[12-20] Defines a function to set all standard deviations proportional to the parameter values. We use this function in the following to set the standard deviation to zero everywhere. 

\item[23-29] Parameters of the analysis. $N$ denotes the resolution of the parameter we vary, and $runs$ the number of iterations, i.e. a total of $N\cdot runs$ simulations are performed. In L29 we define the insertion angle to be varied linearly between 0 and $\pi/2$.

\item[33-52] Definition of a function that performs the simulation and returns mean root tip depth and radial distance. First we create a root system and set the standard deviation to zero L34-L36. L37, amd L39 sets the insertion angle. L41 initializes the root system and states that basal root are of root type 1 (same as tap root). The False value turns of verbosity to avoid any outputs to the console. L42 performs the simulation. L44-L51 calculates the mean root tip depth and radial distance. 

\item[55-73] This section performs the computation of all simulation runs. L55-56 preallocates the resulting arrays. L62-L68 performs the parallel computations, index $i$ is the index of the insertion angle. L71-73 calculates the mean values per simulation run.

\item[74-83] Creates the resulting Figure \ref{fig:sensitivity}. Note that the resulting curves become smoother, if the number of runs is increased (L28).

\end{itemize}

% 
% \begin{figure}
% \centering
% \includegraphics[width=0.7\textwidth]{example_4b.png}
% \caption{Sensitivity of mean root tip depth (left) and radial distance (right) to the insertion angle theta (Example 2f) } \label{fig:sensitivity}
% \end{figure}


\subsection{Interaction Feedback} \label{ssec:interaction}
% 
Root growth is strongly influenced by pedo-climatic conditions, and plant internal state. CPlantBox offers 'build in' ways to develop such models, see \cite{schnepf2018crootbox}. 

CPlantBox is a bottom up model were root growth is first defined under perfect conditions. Adding mechanisms to take environmental conditions into account will alter the root system development by impeding root growth, and by changing soil allocation by roots due to root tropism and due to altered branching patterns.

In this section we assume static soil conditions, and demonstrate the predefined ways how the soil can affect root growth.
Dynamic soil conditions are described in the following Section \ref{s:coupling}. 

Implemented root responses are the change in direction of the growing root tip, as described in previous Section \ref{sec:tropism}.
Further root responses are 
\begin{itemize}
 \item scaling of the elongation rate 
 \item change of insertion angle
 \item change of lateral emergence probability
\end{itemize}

\subsubsection*{Scaling the elongation rate} \label{sec:elongation}

Root elongation rate is influenced by soil properties such as water content, temperature, soil density, solutes, and many more. Regarding the processes that are investigated, various models can be applied. In CPlantBox the elongation rate is scaled with no predefined interpretation, i.e. we have to define a elongation rate scaling, which is dependent on such soil properties. The following example defines two compartments (one left, one right), where we change this scaling, and then analyse the results. The same procedure will be used in Example \ref{sec:insertion_angle} and \ref{sec:branching}.

\lstinputlisting[firstline=1, language=Python, caption=Example 5a]{examples/example5a_elongation.py}

\begin{itemize}

\item[7-10] Creates the root system and opens the parameter file.

\item[13-17] We create a confining box with two overlapping boxes called $left$ and $right$. These geometries are used for later analysis.

\item[20-24] We define static soil properties using SDF (L23, L24) as we did in Section \ref{sec:hydro}. 
The left compartment has the value $minS$, the right $maxS$, between them is a linear gradient of length $slope$. 

\item[27-31] Sets the scaling functions. L29 adjusts axial resolution and L30 tortuosity $sigma$. L31 sets the scale elongation function $f_{se}$ to the soil property (i.e. scales to $minS$ in the left, $maxS$ in the right compartment). 

\item[34-39] Initialization and simulation loop. In a dynamic setting, $soilprop$ needs to be updated in each time step (comment L38).

\item[42-50] Analysis the root length in the left and right compartment. With parameters $minS$ and $slope$ only approximately $21\%$ are located in the left compartment.

\item[53, 54] Writes the results for Paraview visualization (see Figure \ref{fig:elongation}).

\item[57] A vtk simulation of root lengths. Press 'y' to obtain a x-z view of the root system to better see the effect. 

\end{itemize}
 
Next, we give a short layout, how the code would look like, if we take measured data (e.g. density versus depth), and include it in the simulation. 

\lstinputlisting[firstline=1, language=Python, caption=Example 5b]{examples/example5b_scaleelongation.py}

\begin{itemize}

\item[12-16] In L12 an EquidistantGrid1D is created, which is a specialisation from SoilLookUp (in soil.hh). It represents a 1D grid from 0 cm to -100 cm as soil with 100 layers. Additionally scalar data are attached to the grid, L13, L14 we create example soil strength data. From this data we calculate the elongation scales (L15), and set it as grid data (L16). 

\item[L17, L18] Retrieve the elongation scale data at two points. The data is given per layer, no data interpolation is performed.

\item[20, 21] Sets the elongation scaling function to all root types.

\item[25-32] Subsection \ref{ssec:animation} explains how to make an animation in Paraview. Here we use another (slower) approach to create the animation (and a preview) directly in vtk. For this we create an AnimateRoots object (L26), choose the domain size, and start the rendering (L32). 

\item[34-50] Simulation loop. If soil strength changes, the elongation scales must be calculated again (L32), and attached to the grid (L35). The frames of the animations are written in L50. While the approach is convenient, it is rather slow since for each frame a SegmentAnalyser object is created and rendered from scratch which is not feasible for large root systems. 

\item[52,53] Exports results as vtp, and creates a vtk plot. The effect of the dense layer can hardy be seen in the results (between -10 and -15 cm depth, laterals will be shorter). But the created animation reveals the effect (see exported video example5b.ogv). 

\end{itemize}

% \begin{figure}
% \centering
% \includegraphics[width=0.5\textwidth]{example5a.png}
% \caption{Root elongation rate} \label{fig:elongation}
% \end{figure}


\subsubsection*{Change of insertion angle} \label{sec:insertion_angle}

Nutrient concentration influences the angle between parent roots and laterals. % todo ref
Analog to Example 5a two compartments are created, and the insertion angle is scaled accordingly.

\lstinputlisting[firstline=1, language=Python, caption=Example 5c]{examples/example5c_insertionangle.py}

\begin{itemize}

\item[27-32] Sets the insertion angle scaling functions to second order laterals only (L29). L20 adjusts axial resolution and, L31 tortuosity $sigma$, and L32 sets the scale insertion angle function $f_{sa}$ to the soil property. Additionally, the maximal length of second order roots is redoubled. 

\item[34-39] Initialization and simulation loop.

\item[42-50] Analysis the root insertion angle in the left and right compartment. 

\item[53, 54] Writes the results for Paraview visualization.

\item[57] A vtk simulation of root lengths. Press 'y' to obtain a x-z view of the root system to better see the effect (see Figure \ref{fig:insertion}). 

\end{itemize}

% \begin{figure}
% \centering
% \includegraphics[width=0.5\textwidth]{example5c.png}
% \caption{Insertion angle} \label{fig:insertion}
% \end{figure}


\subsubsection*{Change of lateral emergence probability} \label{sec:branching}

Soil properties can affect branching patterns. In the following example two compartments are created ( analog to Example 5a), and the branching probability is modified in each of them.

\lstinputlisting[firstline=1, language=Python, caption=Example 5d]{examples/example5d_branching.py}

\begin{itemize}

\item[27-30] Adjusts axial resolution and tortuosity.

\item[33, 34] We adjust the inter lateral distances by making it smaller for a factor five.

\item[35, 36] We set the branching probability scaling for the second order laterals. The scaling value means the probability that the branch occurs per day, i.e. 1 means the laterals always emerge (left compartment), or that they emerge with a chance of 0.2 \% per day (right compartment). 

\item[46-54] Analysis the root insertion angle in the left and right compartment. 

\item[61] A vtk simulation of root lengths. Press 'y' to obtain a x-z view of the root system to better see the effect (see Figure \ref{fig:probability}). 

\end{itemize}

Note that a scaling of zero means, that the laterals do never emerge, one means they always do. While the branching probability model is limited, it is easy to modify it to implement plant systemic responses. For this a suitable SoilLookUP must be defined and the method getValue(x, organ) must be overwritten, implementing the plant control of the branching probabilities. 

% \begin{figure}
% \centering
% \includegraphics[width=0.5\textwidth]{example5d.png}
% \caption{Branching density (probabilistic model)} \label{fig:probability}
% \end{figure}


\subsection{RSML} \label{ssec:rsml}
% \input{latex/topics_rsml.tex}



\newpage
\section{CPlantbox Structural Model} \label{sec:cplantobx}

In this section we describe the underlying strutural model of CPlantBox. 

\subsection{General plant organs} \label{ssec:organs}

\subsection{Seed} \label{ssec:seed}

\subsection{Stem} \label{ssec:stem}

\subsection{Leaf} \label{ssec:leaf}

\subsection{Root} \label{ssec:root}



\newpage
\section{Plant Structural Functional Models} \label{sec:fspm}

\subsection{Plant hydraulics} \label{ssec:hydraulics}

\subsection{Benchmark examples M3.1 and M3.2} \label{ssec:benchmarks}

\subsection{The standard uptake fraction ($SUF$) and root system conductivity ($K_{rs}$)} \label{ssec:suf}

\subsection{Plant structure mapped to a macrogrid} \label{ssec:coupling}

\subsection{Model coupling} \label{ssec:coupling}



\newpage
\section{Contributing} \label{sec:contributing}

\subsection{Coding style} \label{ssec:coding_style}

\subsection{Todos} \label{ssec:todos}




\newpage
\bibliographystyle{apalike} 
\bibliography{latex/references} 


\end{document}
